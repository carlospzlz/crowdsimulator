%CROWD SIMULATOR
\documentclass{beamer}

%\usepackage{geometry}
%\geometry{verbose,letterpaper}
%\usepackage{graphicx}
%\usepackage{movie15}

\usepackage[utf8]{inputenc}
%\usepackage[galician]{babel}

\usepackage{pgf}
\usepackage{color}
\usepackage{multimedia}

% Estos son los temas básicos (hay variantes 
% en cada uno de ellos). Descomenta uno y compila.
%\usetheme{Goettingen}
\usetheme{Warsaw}
%\usetheme{Dresden}
%\usetheme{Singapore}
%\usetheme{Szeged}
%\usetheme{Pittsburgh}
%\usetheme{Malmoe}
%\usetheme{Montpellier}


% Algunos temas de colores estándar (son feos!)
%\usecolortheme{albatross}
%\usecolortheme{beetle}
%\usecolortheme{crane}
%\usecolortheme{wolverine}

% Este comando hace los puntos redonditos
\beamertemplateballitem

% Esta opción sirve para que, cuando pasamos la
% presentación con pausas, lo que esté oculto 
% se vea de color muy clarito
\setbeamercovered{transparent}


% Declaración de imágenes a usar 
%\pgfdeclareimage[opciones]{etiqueta}{path+fichero}
\pgfdeclareimage[width=60pt]{bu_logo}{images/bu_logo}
\pgfdeclareimage[width=60pt]{ncca_logo}{images/ncca_logo_inverted}
\pgfdeclareimage[width=140pt]{droids}{images/droids}
\pgfdeclareimage[width=140pt]{matrix}{images/matrix}
\pgfdeclareimage[width=110pt]{zombies}{images/zombies}
\pgfdeclareimage[width=150pt]{battlefield}{images/battlefield}
\pgfdeclareimage[width=300pt]{navigation_fields}{images/navigation_fields}
\pgfdeclareimage[width=250pt]{reynolds_flocking}{images/reynolds_flocking}
\pgfdeclareimage[width=75pt]{cohesion}{images/cohesion}
\pgfdeclareimage[width=75pt]{separation}{images/separation}
\pgfdeclareimage[width=75pt]{alignment}{images/alignment}
\pgfdeclareimage[width=300pt]{rohan_army}{images/rohan_army}
\pgfdeclareimage[width=50pt]{lego_cartoon}{images/lego_cartoon}
\pgfdeclareimage[width=50pt]{body}{images/body}
\pgfdeclareimage[width=50pt]{brain}{images/brain}
\pgfdeclareimage[width=170pt]{agent}{images/agent}
\pgfdeclareimage[width=100pt]{lua_script}{images/lua_script}
\pgfdeclareimage[width=100pt]{lua_script_statistics}{images/lua_script_statistics}
\pgfdeclareimage[width=100pt]{lua_script_neural_network}{images/lua_script_neural_network}
\pgfdeclareimage[width=100pt]{lua_script_fsm}{images/lua_script_fsm}
\pgfdeclareimage[width=100pt]{messages}{images/messages}


% Título, subtítulo, autor, afiliación y fecha
\title[Crowd Simulation]{Crowd Simulation based on Emergent Behaviours}
%\subtitle{\scriptsize{Directores: Manuel F. González Penedo, Marcos Ortega Hortas}}
\author{Carlos Pérez López}
\institute{MSc Computer Animation and Visual Effects}
\date{27, August 2013}

\begin{document}

% Portada
\begin{frame}
  \begin{center}
    \begin{tabular}{c c c}
	  \pgfuseimage{bu_logo} & & \pgfuseimage{ncca_logo} \\
	  \scriptsize{Bournemouth University} & & \scriptsize{National Centre of Computer Animation} \\
    \end{tabular}
  \end{center}
	\titlepage
\end{frame}

% index
\begin{frame}
	% Con este comando se pone el título de la diapositiva
	\frametitle{Index}
	\tableofcontents[hideallsubsections]
\end{frame}

\AtBeginSection[]
{
  \begin{frame}<beamer>
    \frametitle{Index}
    \tableofcontents[current,currentsubsection]
  \end{frame}
}

\AtBeginSubsection[]
{
  \begin{frame}<beamer>
    \frametitle{Index}
    \tableofcontents[current,currentsubsection]
  \end{frame}
}

\section{Introduction}

\subsection{Motivation}

\begin{frame}
	\frametitle{Crowd in Films}
     \begin{center}
       \begin{tabular}{cc}
 	 	\pgfuseimage{droids} & \pgfuseimage{matrix} \\
	    \pgfuseimage{zombies} & \pgfuseimage{battlefield}\\
      \end{tabular}
     \end{center}
\end{frame}

\begin{frame}
\frametitle{Crowd in Films}
  \begin{columns}
    \column{5cm}
     \begin{exampleblock}{Contributions}
      \begin{itemize}  
	\item Visually stunning
	\item Catch public's attention
	\item Enrich story
	\item The story may just need it
      \end{itemize}
     \end{exampleblock}
     \column{5cm}
  \end{columns}
\end{frame}

\begin{frame}
\frametitle{Crowd in Films}
  \begin{columns}
    \column{5cm}
     \begin{exampleblock}{Contributions}
      \begin{itemize}  
	\item Visually stunning
	\item Catch public's attention
	\item Enrich story
	\item The story may just need it
      \end{itemize}
     \end{exampleblock}
     \column{5cm}
     \begin{alertblock}{Issues}
      \begin{itemize}
    \item Requires either real extra cast
    \item or more animators.
	\item Requires more time and money
      \end{itemize}
     \end{alertblock}
  \end{columns}
\end{frame}

\subsection{Related Work}

\begin{frame}
\frametitle{Motion Planning from Crowd}
\begin{center}
	\pgfuseimage{navigation_fields}
\end{center}
\end{frame}

\begin{frame}
\frametitle{Crowd Motion Simulation}
\begin{center}
	\pgfuseimage{reynolds_flocking}
\end{center}
\end{frame}

\begin{frame}
\frametitle{C. Reynold's Flocking Model}
\begin{center}
\begin{tabular}{ccc}
	\pgfuseimage{cohesion} & \pgfuseimage{separation} & \pgfuseimage{alignment} \\
	Cohesion & Separation & Alignment \\
\end{tabular}
\end{center}
\end{frame}

\begin{frame}
\frametitle{MASSIVE Software}
\begin{center}
	\pgfuseimage{rohan_army}
\end{center}
\end{frame}

\section{Agent-Based Model}

\begin{frame}
\frametitle{Agent}
\begin{center}
	\pgfuseimage{lego_cartoon}
\end{center}
\end{frame}

\begin{frame}
\frametitle{Agent}
\begin{center}
\begin{tabular}{c}
	\pgfuseimage{brain}\\ \pgfuseimage{lego_cartoon}\\
\end{tabular}
\end{center}
\end{frame}

\subsection{Agent Body}

\begin{frame}
\frametitle{Agent Body}
\begin{center}
\begin{columns}
	\column{4cm}
	\begin{block}{Physical Properties}
	\begin{itemize}
	\item Mass
	\item Strength
	\item Maximum Strength
	\item Velocity
	\item Maximum Speed
	\item Vision Radius
	\end{itemize}
	\end{block}
	\column{6cm}
	\pgfuseimage{agent}
\end{columns}
\end{center}
\end{frame}

\subsection{Agent Brain}

\begin{frame}
\frametitle{Agent Brain}
\begin{center}
\begin{columns}
	\column{4cm}
	\begin{block}{Behaviour}
	The brain receives information about the agent body, the environment
	and the interactions with other agents. After processing, it determines
	which actions the agent must perform.
	\end{block}
	\column{3cm}
	\pgfuseimage{lua_script}
\end{columns}
\end{center}
\end{frame}

\begin{frame}
\frametitle{Agent Brain}
\begin{center}
\begin{columns}
	\column{4cm}
	\begin{block}{Behaviour}
	The brain receives information about the agent body, the environment
	and the interactions with other agents. After processing, it determines
	which actions the agent must perform.
	\end{block}
	\column{3cm}
	\pgfuseimage{lua_script_statistics}
\end{columns}
\end{center}
\end{frame}

\begin{frame}
\frametitle{Agent Brain}
\begin{center}
\begin{columns}
	\column{4cm}
	\begin{block}{Behaviour}
	The brain receives information about the agent body, the environment
	and the interactions with other agents. After processing, it determines
	which actions the agent must perform.
	\end{block}
	\column{3cm}
	\pgfuseimage{lua_script_neural_network}
\end{columns}
\end{center}
\end{frame}

\begin{frame}
\frametitle{Agent Brain}
\begin{center}
\begin{columns}
	\column{4cm}
	\begin{block}{Behaviour}
	The brain receives information about the agent body, the environment
	and the interactions with other agents. After processing, it determines
	which actions the agent must perform.
	\end{block}
	\column{3cm}
	\pgfuseimage{lua_script_fsm}
\end{columns}
\end{center}
\end{frame}

\subsection{Interactions Among Agents. Message Passing}

\begin{frame}
\frametitle{Interactions Among Agents}
\begin{center}
\begin{columns}
	\column{4cm}
	\begin{block}{Message}
	\begin{itemize}
	\item Agent Identification
	\item Label
	\item Position
	\item Strength
	\end{itemize}
	\end{block}
	\column{3cm}
	\pgfuseimage{messages}
\end{columns}
\end{center}
\end{frame}

\section{Crowd Engine}

\subsection{Handling Large Amounts of Agents. Space Partition}

\begin{frame}
\frametitle{Searching Neighbours in Vision Radius Algorithm}
\begin{center}
\begin{columns}
	\column{4cm}
	\pgfuseimage{lua_script_fsm}
	\column{4cm}
	\begin{block}{Two procedures}
	\begin{itemize}
	\item Find the agents in the cells which
	include the vision radius
	\item Find the neighbours from those agents
	\end{itemize}
	\end{block}
\end{columns}
\end{center}
\end{frame}

\begin{frame}
\frametitle{Searching Neighbours in Vision Radius Algorithm}
\begin{center}
\begin{columns}
	\column{4cm}
	\pgfuseimage{lua_script_fsm}
	\column{4cm}
	\begin{block}{Heuristic}
	``two agents that are spatially close 
	 may share many common neighbours'' (Lee, 2010)
	\end{block}
\end{columns}
\end{center}
\end{frame}

\subsection{Virtual Force Model}

\section{Results}

\begin{frame}
\frametitle{Shooter Droid}
\begin{center}
	\pgfuseimage{lego_cartoon}
\end{center}
\end{frame}

\begin{frame}
\frametitle{Shooter Droid}
\begin{center}
	\pgfuseimage{lego_cartoon}
\end{center}
\end{frame}

\begin{frame}
\frametitle{Shooter Droid}
\begin{center}
	\pgfuseimage{lego_cartoon}
\end{center}
\end{frame}

\section{Conclusion}

\subsection{Drawbacks}

\subsection{Future Work}

\end{document}